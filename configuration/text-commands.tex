% Commands to be used in the text.

% LTEX: enabled=false

% Scene breaks are, as is customary, represented by vertical space between
% paragraphs (and no indentation on the next).
% Now, there’s a problem with this: If the scene break happens to fall at
% the very end of a page, it’s (especially in digital type systems such as
% HTML) liable to be turned into a page break, erasing most all
% indication that the scene was switched and thus confusing the reader.
% We solve this by adding a blank line to the *top of the next page*
% in that case.
% There’s a secondary edge case as well: In the case that your scene break
% falls *one line before* the end of the page, this would normally mean
% that a line is left blank at the bottom, and the next page starts flush
% with the top. This is also confusing, as it’s harder to spot the missing
% line at the bottom than it is at the top – so we ensure that the line
% break always sticks together with the paragraph *after,* not before.
\NewDocumentCommand{\scene}{}{%
    \iftexforht{%
        % This works in KOReader, but on stock Kindle, it just acts as
        % a regular `<br>` – that is to say, it turns into a page break
        % and leaves no trace at the top of the next page. It’s highly
        % unfortunate, but it seems that at the moment, the only reliable
        % way of signaling a scene break in ebooks is to use a dinkus
        % of some description – which I’d rather not, as it changes the
        % reading of the text, so we’re going with this imperfect solution.
        \EndP \IgnorePar \HCode{<p style="page-break-after: avoid; break-after: avoid;"><br></p>} \IgnoreIndent \par%
    }{%
        \Needspace{2\baselineskip}%
        {\penalty0 \vspace*{1\baselineskip}}%
        \NoIndentAfterThis%
    }%
}


% European-style dialogue (“―What in the world? said Moomin.”).
% Set to the `U+2015 Horizontal Bar` instead of em dash.
\NewDocumentCommand{\s}{m}{\ifvmode{}\else{\par}\fi―#1}

% Dialogue continuation.
\NewDocumentCommand{\cont}{m}{―#1}


% Cited titles. “Title italic.”
% Takes an optional argument which is a two-letter ISO language code
% that will show up in the HTML. For example, `\titalic{ja}{Otoko
% wa Tsurai yo}` will render as `<cite lang="ja-Latn">Otoko wa Tsurai
% yo</cite>`.
% Acts as italics in the PDF version.
\NewDocumentCommand{\titalic}{o m}{%
    \iftexforht{%
        \HCode{<cite%
            \IfNoValueF{#1}{\space lang="#1-Latn"}%
        >}#2\HCode{</cite>}%
    }{%
        \textit{#2}%
    }%
}


% Circled text.
% (This one requires `\usepackage{circledsteps}`, which is why it’s
% not uncommented by default.)
% \NewDocumentCommand{\circled}{mm}{%
%     \iftexforht%
%         {#2}%
%         {\smash{\smaller[1]{\Circled{#1}}}}%
% }


% Squared text.
\NewDocumentCommand{\squared}{m}{%
    \iftexforht%
        {\HCode{<span class="squared">}#1\HCode{</span>}}%
        {\setlength{\fboxsep}{1.5pt}\fbox{\smaller[1]{#1}}}%
}


% Transliterations.
% The first argument is a two-letter ISO language code, and the
% second the text. For example, `\translit{fr}{joie de vivre}`
% renders as `<i lang="fr-Latn">joie de vivre</i>`.
% Acts as italics in the PDF version.
\NewDocumentCommand{\translit}{mm}{%
    \iftexforht%
        {\HCode{<i lang="#1-Latn">}#2\HCode{</i>}}%
        {\textit{#2}}%
}


% Emphasized text (`<em>…</em>`). Acts as italics.
\RenewDocumentCommand{\em}{m}{%
    \iftexforht%
        {\HCode{<em>}#1\HCode{</em>}}%
        {\textit{#1}}%
}


% Italicized text without a semantic meaning (`<i>…</i>`).
\NewDocumentCommand{\italic}{m}{%
    \iftexforht%
        {\HCode{<i>}#1\HCode{</i>}}%
        {\textit{#1}}%
}
