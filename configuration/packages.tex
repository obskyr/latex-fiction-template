% Packages to load at the start of the preamble.
% Not every package needs to be in here. Some may make more sense to import
% in conjunction with other code.

% LTEX: enabled=false


\usepackage{anyfontsize}
% In addition to allowing the use of the `\href` macro, `hyperref`
% also sets useful PDF metadata.
\usepackage[hyperfootnotes=false]{hyperref}
% Prevents breaking words at the end of a page. In principle, I’d be fine
% with breaking a word across the same spread, but I don’t know of a
% way to make that happen.
\usepackage[hyphenation]{impnattypo}
\usepackage{needspace}
\usepackage[asterism]{sectionbreak}
\usepackage{subfiles}

% Change this to change the language settings for your book.
\usepackage[USenglish]{babel}

% Footnotes should, in most cases, be handled differently between the
% PDF version and the textual e-book version.
\iftexforht{
    \RenewDocumentCommand{\thefootnote}{}{\roman{footnote}}
    % Without this, TeX4ht will insert hard-coded size `<span>`s.
    % We handle the size in CSS instead.
    \setkomafont{footnote}{\normalsize}
    % It’s actually meaningful that this second one is set to a non-zero
    % length. It makes TeX4ht add the `indent` class to following paragraphs
    % rather than the `noindent` class.
    \deffootnote{0em}{2em}{\thefootnotemark.}
}{
    \usepackage[perpage,symbol*]{footmisc}
}
